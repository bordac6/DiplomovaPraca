\section{Návrh modelu}\label{sec:proposal}
Návrh nášho nového systému bude založený na regresnom prístupe. Podobne ako v predchádzajúcom riešení \ref{chpt:hourglass} s tým rozdielom, že použijeme 2D konvolúcie a naša sieť bude predikovať 2D tepelné mapy. 2D konvolúcie sú jednoduchšie na výpočet a má menšie nároky na kapacitu pamäte. Teda by sme znížili hardvérové požiadavky na chod systému, zatiaľ čo presnosť predikcie by sa nemusela znížiť.

Budeme predikovať 11 kĺbov na ruke. Sú to dva kĺby na každom prste a zápästie. Vďaka dvom kĺbom vieme určiť pozíciu prsta v obraze a tiež aj smer, ktorým prst ukazuje. Výstupná informácia z nášho systému bude preto vhodná aj pre úlohy, kde robotovi ukážeme na predmet a on bude vedieť označiť takto zvolený predmet.

Pridanie hĺbkovej informácie pre predikovaný kľúčový bod bude robiť mapovaním tepelnej mapy s hĺbkovou mapou, ktorú nám poskytuje výstup z kamery \ref{camera} Inel RealSense D430i. %Pre výber hĺbkových dát, ktorými určíme tretiu súradnicu bodu použijeme úpravu hĺbkových dát podobne ako v kapitole \ref{chap:previous_solutions} časti \ref{chpt:depthBase}. 

Vytvoríme CNN zloženú z dvoch hourglass \ref{architecture_hourglass} modulov. Názorne ilustrovaný návrh je na obr. \ref{img:our_architecture}. Výstup prvého modulu bude vstupom do druhého hourglass modulu, ktorého výstup vytvorí pre každý kĺb jednu tepelnú mapu. Preto vytvoríme zo vstupných súradníc kĺbov tepelnú mapu pre každý kĺb. Mapa bude obsahovať Gausovu distribúciu kladných hodnôt so stredom v súradniciach kĺbu a štandardnou odchýlkou. Výber hodnoty odchýlky bude popísaný ďalej v tejto kapitole. Trénovanie bude potom reprezentované výpočtom euklidovskej vzdialenosti medzi očakávanými a predikovanými tepelnými mapami. Pre každých 32 vstupných obrázkoch bude vypočítaná priemerná hodnota tejto vzdialenosti, podľa ktorej sa upravia váhy v CNN.

Na trénovanie siete použijeme dataset NYUHands popísanom v \ref{datasets} a najlepší model následne dotrénujeme na vlastnom datasete. Súradnice v našich obrázkoch určíme ručne. Vytvoríme si jednoduchý program, ktorý vezme pozíciu kliknutia v obrázku a zaznamená súradnice kĺbu. Na zvýšenie presnosti predikcie kĺbov pre naše konkrétne prostredie bude stačiť menšie množstvo obrázkov, keďže sieť už bude vedieť predikovať nejaké pozície ruky.